\documentclass{beamer}

\usetheme{CambridgeUS}
\title{Assignment 7} 
\author{Akhila Kumbha,CS21BTECH11031}
\date{\today}
\logo{\large \LaTeX{}}


\providecommand{\brak}[1]{\ensuremath{\left(#1\right)}}
\providecommand{\pr}[1]{\ensuremath{\Pr\left(#1\right)}}
\providecommand{\cbrak}[1]{\ensuremath{\left\{#1\right\}}}
\newcommand{\myvec}[1]{\ensuremath{\begin{pmatrix}#1\end{pmatrix}}}
\begin{document}


\begin{frame}
    \titlepage 
\end{frame}

\logo{}

\begin{frame}{Outline}
    \tableofcontents
\end{frame}


\section{Question}
\begin{frame}{Question}

Show that if the random variables $X_i$ are i.i.d. and normal, then their sample mean $\bar{X}$ and sample variances $S^2$ are two independent random variables.\\

\end{frame}


\section{Solution}
\begin{frame}{Solution}
Given that the random variables $X_i$ are i.i.d. and normal.\\
We wish to show that RVs
\begin{align}
    \bar{X}=\frac{1}{n} \sum_{i=1}^{n}X_i\\
    S^2=\frac{1}{n-1} \sum_{i=1}^{n} \brak{{X_i}-\bar{X}}^2
\end{align}
are independent.Since $S^2$ is a function of the n RVs $X_i-\bar{X}$, it suffices to show that each of these RVs is independent of $\bar{X}$.
\end{frame} 

\begin{frame}
We assume for simplicity that $E(X_i)=0$.\\Clearly,\\
\begin{align}
  E(X_i\bar{X})=\frac{1}{n}E\cbrak{X_i^2}=\frac{\sigma^2}{n}\\
  E(\bar{X}\bar{X})=\frac{1}{n^2}\sum_{i=1}^{n} X_i^2=\frac{\sigma^2}{n}
\end{align}
because $E(X_iX_j)=0$ for $i\neq j$.\\Hence,
\begin{align}
    E((X_i-\bar{X})\bar{X})=0
\end{align}
\end{frame}
\begin{frame}
Thus, the RVs $X_i-\bar{X}$ and $\bar{X}$ are orthogonal and since they are jointly normal, they are independent.
\end{frame}

\end{document}